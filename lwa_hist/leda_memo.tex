\documentclass[fleqn, onecolumn]{article}
\usepackage{amsmath}
%\usepackage[landscape]{geometry}
\usepackage{graphicx}
\usepackage{subfig}
\usepackage{fullpage}
%\usepackage{hyperref}
%\usepackage{listings}
%\setlength{\parindent}{0in}

\begin{document}
\title{LEDA Memo: 8 Bit Sampling of LWA Data}
\date{}
\author{Terry Filiba}
\maketitle

In order to determine the suitability of an 8 bit ADC for sampling LWA data, we analyzed data taken every 15 minutes over a 24 hour period from 6 LWA dual pol stands. The data was captured using the currently available 12 bit ADCs.

Initial analysis of the data showed that at least 97\% of the samples lie within the range $[-128,128)$. Without adjusting any gains there would be $< 3\%$ error using an 8 bit ADC.
If the ADC gain is set so that the RMS is $20$ the error rate of the ADC gets even lower. With the sampled data rescaled, saturation only occurs in $<0.001\%$ of the samples using an 8 bit ADC (an error rate of $10^{-5}$). 

\begin{table}[h]
$ \begin{array}{c||c|c|c}  
\text{Stand} & \text{RMS} & \text{samples }\in [-128,128) & \text{samples }\in [-128,128)\text{ with RMS}=20 \\ 
Stand001X & 34.118798 & 99.989641 & 99.999833 \\ 
Stand001Y & 28.544380 & 99.99885 & 99.9999 \\ 
Stand010X & 55.174325 & 98.006408 & 99.999708 \\ 
Stand010Y & 46.112851 & 99.452791 & 99.999925 \\ 
Stand054X & 43.717739 & 99.662691 & 99.999641 \\ 
Stand054Y & 47.889111 & 99.243641 & 99.999916 \\ 
Stand248X & 26.123759 & 99.999583 & 99.99985 \\ 
Stand248Y & 30.707551 & 99.998758 & 99.999991 \\ 
Stand251X & 58.756399 & 97.093866 & 99.999633 \\ 
Stand251Y & 58.047033 & 97.285133 & 99.99965 \\ 
Stand258X & 27.404131 & 99.999291 & 99.999766 \\ 
Stand258Y & 28.380539 & 99.999533 & 100.0
 \end{array}  
$
\end{table}

Histogramming of the data shows that the majority of samples lie within 8 bits. 
The following pages contain plots of the data from each stand-pol. 
The blue bars highlight the samples that could be represented with 8 bits without gain adjustment. 
The yellow bars show the samples that would not saturate if the gain was adjusted so that the RMS is $20$.
The log plots of the data emphasize the short tails in the histogram, showing that adjusting the gain will be an effective way to reduce the error rate. 
\begin{figure}[ht] 				\subfloat{\includegraphics[width=0.5\textwidth]{plots/Stand001Xhist.png}} 				\subfloat{\includegraphics[width=0.5\textwidth]{plots/Stand001Xhist_log.png}} 				\caption{Data from Stand001X. RMS is 34.118799 Samples used : 3792000000. If the RMS is unchanged 99.989642 percent of the samples will lie within [-128,128).  				 With an RMS of 20 99.999833 percent of the samples will lie within [-128,128).} 				\end{figure} 

\begin{figure}[ht] 				\subfloat{\includegraphics[width=0.5\textwidth]{plots/Stand001Yhist.png}} 				\subfloat{\includegraphics[width=0.5\textwidth]{plots/Stand001Yhist_log.png}} 				\caption{Data from Stand001Y. RMS is 28.544381 Samples used : 3792000000. If the RMS is unchanged 99.998850 percent of the samples will lie within [-128,128).  				 With an RMS of 20 99.999900 percent of the samples will lie within [-128,128).} 				\end{figure} 

\begin{figure}[ht] 				\subfloat{\includegraphics[width=0.5\textwidth]{plots/Stand010Xhist.png}} 				\subfloat{\includegraphics[width=0.5\textwidth]{plots/Stand010Xhist_log.png}} 				\caption{Data from Stand010X. RMS is 55.174325 Samples used : 3792000000. If the RMS is unchanged 98.006408 percent of the samples will lie within [-128,128).  				 With an RMS of 20 99.999708 percent of the samples will lie within [-128,128).} 				\end{figure} 

\begin{figure}[ht] 				\subfloat{\includegraphics[width=0.5\textwidth]{plots/Stand010Yhist.png}} 				\subfloat{\includegraphics[width=0.5\textwidth]{plots/Stand010Yhist_log.png}} 				\caption{Data from Stand010Y. RMS is 46.112851 Samples used : 3792000000. If the RMS is unchanged 99.452792 percent of the samples will lie within [-128,128).  				 With an RMS of 20 99.999925 percent of the samples will lie within [-128,128).} 				\end{figure} 

\begin{figure}[ht] 				\subfloat{\includegraphics[width=0.5\textwidth]{plots/Stand054Xhist.png}} 				\subfloat{\includegraphics[width=0.5\textwidth]{plots/Stand054Xhist_log.png}} 				\caption{Data from Stand054X. RMS is 43.717739 Samples used : 3792000000. If the RMS is unchanged 99.662692 percent of the samples will lie within [-128,128).  				 With an RMS of 20 99.999642 percent of the samples will lie within [-128,128).} 				\end{figure} 

\begin{figure}[ht] 				\subfloat{\includegraphics[width=0.5\textwidth]{plots/Stand054Yhist.png}} 				\subfloat{\includegraphics[width=0.5\textwidth]{plots/Stand054Yhist_log.png}} 				\caption{Data from Stand054Y. RMS is 47.889111 Samples used : 3792000000. If the RMS is unchanged 99.243642 percent of the samples will lie within [-128,128).  				 With an RMS of 20 99.999917 percent of the samples will lie within [-128,128).} 				\end{figure} 

\begin{figure}[ht] 				\subfloat{\includegraphics[width=0.5\textwidth]{plots/Stand248Xhist.png}} 				\subfloat{\includegraphics[width=0.5\textwidth]{plots/Stand248Xhist_log.png}} 				\caption{Data from Stand248X. RMS is 26.123760 Samples used : 3792000000. If the RMS is unchanged 99.999583 percent of the samples will lie within [-128,128).  				 With an RMS of 20 99.999850 percent of the samples will lie within [-128,128).} 				\end{figure} 

\begin{figure}[ht] 				\subfloat{\includegraphics[width=0.5\textwidth]{plots/Stand248Yhist.png}} 				\subfloat{\includegraphics[width=0.5\textwidth]{plots/Stand248Yhist_log.png}} 				\caption{Data from Stand248Y. RMS is 30.707551 Samples used : 3792000000. If the RMS is unchanged 99.998758 percent of the samples will lie within [-128,128).  				 With an RMS of 20 99.999992 percent of the samples will lie within [-128,128).} 				\end{figure} 

\begin{figure}[ht] 				\subfloat{\includegraphics[width=0.5\textwidth]{plots/Stand251Xhist.png}} 				\subfloat{\includegraphics[width=0.5\textwidth]{plots/Stand251Xhist_log.png}} 				\caption{Data from Stand251X. RMS is 58.756399 Samples used : 3792000000. If the RMS is unchanged 97.093867 percent of the samples will lie within [-128,128).  				 With an RMS of 20 99.999633 percent of the samples will lie within [-128,128).} 				\end{figure} 

\begin{figure}[ht] 				\subfloat{\includegraphics[width=0.5\textwidth]{plots/Stand251Yhist.png}} 				\subfloat{\includegraphics[width=0.5\textwidth]{plots/Stand251Yhist_log.png}} 				\caption{Data from Stand251Y. RMS is 58.047033 Samples used : 3792000000. If the RMS is unchanged 97.285133 percent of the samples will lie within [-128,128).  				 With an RMS of 20 99.999650 percent of the samples will lie within [-128,128).} 				\end{figure} 

\begin{figure}[ht] 				\subfloat{\includegraphics[width=0.5\textwidth]{plots/Stand258Xhist.png}} 				\subfloat{\includegraphics[width=0.5\textwidth]{plots/Stand258Xhist_log.png}} 				\caption{Data from Stand258X. RMS is 27.404131 Samples used : 3792000000. If the RMS is unchanged 99.999292 percent of the samples will lie within [-128,128).  				 With an RMS of 20 99.999767 percent of the samples will lie within [-128,128).} 				\end{figure} 

\begin{figure}[ht] 				\subfloat{\includegraphics[width=0.5\textwidth]{plots/Stand258Yhist.png}} 				\subfloat{\includegraphics[width=0.5\textwidth]{plots/Stand258Yhist_log.png}} 				\caption{Data from Stand258Y. RMS is 28.380539 Samples used : 3792000000. If the RMS is unchanged 99.999533 percent of the samples will lie within [-128,128).  				 With an RMS of 20 100.000000 percent of the samples will lie within [-128,128).} 				\end{figure} 



\end{document}